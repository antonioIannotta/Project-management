\section*{Introduzione}
\subsection*{Smart Parking}
Il progetto preso in considerazione riguarda lo sviluppo di un'applicazione per la gestione intelligente di un parcheggio. Prima di procedere con la descrizione 
dell'approccio utilizzato per la gestione del progetto è opportuno presentare il dominio all'interno del quale questa applicazione andrebbe a porsi e l'idea di realizzazione della stessa.
\subsubsection*{Dominio}
L'enorme numero di autoveicoli a disposizione dei cittadini da una parte e le sempre più numerose regolamentazioni che vengono messe in atto dalle città, unite con una crescente possibilità di entrate per i comuni, ha fatto si che un numero sempre maggiore di parcheggi sia a pagamento. \\
All'interno di questo contesto, considerare i vari slot presenti all'interno dii una certa località geografica come un singolo parcheggio con il quale gli utenti possono interagire nel senso di visualizzazione dello stato dei vari slot che compongono il parcheggio così come nella gestione temporale delle varie soste.
\subsubsection*{Idea di realizzazione}
L'obiettivo è quello di realizzare un'applicazione che sia in grado di consentire agli utenti di interagire con il parcheggio, visualizzato in formato grafico, attraverso la visualizzazione dello stato dei vari slot che compongono il parcheggio, attraverso la possibilità di selezionare un certo slot per la sosta selezionandone la durata, incrementandola eventualmente ed infine liberare lo slot.